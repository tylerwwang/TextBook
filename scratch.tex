Boolean algebra is the formal treatment of manipulating expressions that behave like evaluation true false statements.
I could immediately list the axioms\footnotemark that define Boolean algebra, but I find this doesn't build the intuition for the motivation of such a construction.
Instead, let's take a journey in uncovering some of the properties we want our system to have to adequately describe what we want to describe.

\footnotetext{It's better to interpret an axiom here as a given set of rules assumed to be true in a mathematical system.
In a later section, the way these rules may emerge as axioms by our previous definition will be made clearer.}

\subsection{Defining Boolean Algebra}
Now that we have analyzed our simple system of true/false statements, it's time to turn our attention to how this can motivate Boolean algebra.

We would like our system to retain the properties of that our true/false statements observe, so we can interpret our system of true/false statements as a Boolean algebra.
Now I must note here, while our true/false statement system is a boolean algebra, not \textit{all} Boolean algebras can be intereted as a true/false statement system.
Boolean algebras are more general and can be scaled up to contain arbitrary elements.
If this doesn't yet make sense, we will expand on it after we define our Boolean algebra.

\begin{define}[Boolean Algerba]
	Let $\mathcal{B}$ be a set. Then $(\mathcal{B}, \vee, \wedge, \neg)$ is a Boolean algebra if $\vee, \wedge, \neg$ are stable in $\mathcal{B}$ and the following are satified:
		\begin{longtable}{lV{8em}V{8em}}
			\textit{(a1)} \textbf{(Associative)} & 
				$p \vee (q \vee r) = (p \vee q ) \vee r,$ &
				$p \wedge (q \wedge r) = (p \wedge q ) \wedge r$ \\
		
			\textit{(a2)} \textbf{(Commutative)} &
				$p \vee q = q \vee p,$ &
				$p \wedge q = q \wedge p$ \\
			
			\textit{(a3)} \textbf{(Distributive)} &
				$p \wedge (q\vee r) = (p \wedge q) \vee (p \wedge r),$ &
				$p \vee (q\wedge r) = (p \vee q) \wedge (p \vee r),$\\
			
			\textit{(a4)} \textbf{(Identity)} &
				$p \vee 0 = p,$ &
				$p \wedge 1 = p$ \\
				
			\textit{(a5)} \textbf{(Complement)} &
				$p \vee \neg p = 0,$ &
				$p \wedge \neg p = 1$ \\
				
			\textit{(a6)} \textbf{(Absorbtion)} &
				$p \vee (p \wedge q) = p,$ &
				$p \wedge (p\vee q) = p$
		\end{longtable}
\end{define}

Now, if you notice, I've replaced the symbol T with 0, and F with 1.
I made these substitutions to highlight the fact that a Boolean algebra is not the same as our true/false statement system.
