\usepackage{amssymb}
\usepackage{amsmath, esint}
\usepackage{amsthm}
\usepackage{array}
\usepackage{longtable}
\usepackage{varwidth}
\usepackage{mathtools}
\usepackage{pdfpages}
\usepackage{fancyhdr}
\usepackage[figurename=Figure]{caption}
\usepackage{empheq}
\usepackage{mdframed}
\usepackage[b5paper, left=1.25in,right=1.25in,top=0.5in,bottom=0.5in,footskip=0.75in,includeheadfoot]{geometry}
\usepackage{tikz}
\usepackage{pgfplots}
\usepackage{hyperref}
\usepackage{wrapfig}
\usepackage{enumitem}
\usepackage{lastpage}
\usepackage{zref-totpages}
\usepackage{setspace}
\usepackage{emptypage}
\usepackage{multirow}
\usepackage{enumitem}

\usetikzlibrary{calc, backgrounds,angles, quotes}

\tikzset{
    partial ellipse/.style args={#1:#2:#3}{
        insert path={+ (#1:#3) arc (#1:#2:#3)}
    }
}

\theoremstyle{definition}

\newmdtheoremenv{theorem}{Theorem}
\newmdtheoremenv{lemma}[theorem]{Lemma}
\newmdtheoremenv{prop}[theorem]{Proposition}
\newmdtheoremenv{define}[theorem]{Definition}
\newmdtheoremenv{thm}[theorem]{Theorem}
\newmdtheoremenv{cor}[theorem]{Corollary}

\numberwithin{equation}{chapter}
\numberwithin{theorem}{chapter}
\numberwithin{figure}{chapter}

\newenvironment{ex}[0]{
	\refstepcounter{theorem}
	\par
	\noindent
    \textbf{Example \thetheorem.}}
    {\hfill$\spadesuit$
    \par}
\newenvironment{rem}[0]{
	\refstepcounter{theorem}
	\par
	\noindent
    \textbf{Remark \thetheorem.}}
    {\par}

\newenvironment{block}
    {\begin{center}\begin{em}\small
    \begin{tabular}{p{0.9\linewidth}}
    }
    { 
    \end{tabular}
    \end{em}
    \end{center}
    }

\def\real{\mathbb{R}}
\def\complex{\mathbb{C}}
\def\rat{\mathbb{Q}}
\def\nat{\mathbb{N}}
\def\integ{\mathbb{Z}}
\def\mod#1{\mathbb{Z}_{#1}}
\def\cpx{\mathbb{C}}

\def\paren#1{\left(#1\right)}
\def\sbrak#1{\left[#1\right]}
\def\cbrak#1{\left\{#1\right\}}

\def\ceil#1{\left\lceil #1 \right\rceil}
\def\floor#1{\left\lfloor #1 \right\rfloor}
\def\abs#1{\left\lvert #1 \right\rvert}
\def\abrak#1{\left\langle #1 \right\rangle}
\def\bra#1{\left\langle #1 \right\rvert}
\def\ket#1{\left\lvert #1 \right\rangle}
\def\braket#1#2{\left\langle #1 \left.\right\lvert #2 \right\rangle}

\def\jand{\quad\text{and}\quad}
\def\jor{\quad\text{or}\quad}
\def\for{\quad\text{for }\,}

\def\ieval#1#2{\Bigg|^{#2}_{#1}}
\def\deval#1{\bigg|_{#1}}

\def\diff#1#2{\frac{d#1}{d#2}}
\def\pdiff#1#2{\frac{\partial#1}{\partial#2}}











