\usepackage{amssymb}
\usepackage{amsmath, esint}
\usepackage{amsthm}
\usepackage{array}
\usepackage{longtable}
\usepackage{varwidth}
\usepackage{mathtools}
\usepackage{pdfpages}
\usepackage{fancyhdr}
\usepackage[figurename=Figure]{caption}
\usepackage{empheq}
\usepackage{mdframed}
\usepackage[b5paper, left=1.25in,right=1.25in,top=0.9in,bottom=0.5in,footskip=0.75in,includeheadfoot]{geometry}
\usepackage{tikz}
\usepackage{tikz-cd}
\usepackage{pgfplots}
\usepackage{hyperref}
\usepackage{wrapfig}
\usepackage{enumitem}
\usepackage{setspace}
\usepackage{emptypage}
\usepackage{multirow}
\usepackage{afterpage}
\usepackage{cleveref}
\usepackage{aliascnt}

\newcommand{\blankpage}{%
    \null
    \thispagestyle{empty}%
    \addtocounter{page}{-1}%
    \newpage}

\usetikzlibrary{calc, backgrounds,angles, quotes}
\pgfplotsset{compat=1.15}

\tikzset{
    partial ellipse/.style args={#1:#2:#3}{
        insert path={+ (#1:#3) arc (#1:#2:#3)}
    }
}

\newcommand{\real}{\mathbb{R}}
\newcommand{\complex}{\mathbb{C}}
\newcommand{\rat}{\mathbb{Q}}
\newcommand{\nat}{\mathbb{N}}
\newcommand{\integ}{\mathbb{Z}}
\newcommand{\cpx}{\mathbb{C}}

\newcommand{\paren}[1]{\left(#1\right)}
\newcommand{\sbrak}[1]{\left[#1\right]}
\newcommand{\cbrak}[1]{\left\{#1\right\}}

\newcommand{\ceil}[1]{\left\lceil #1 \right\rceil}
\newcommand{\floor}[1]{\left\lfloor #1 \right\rfloor}
\newcommand{\abs}[1]{\left\lvert #1 \right\rvert}
\newcommand{\abrak}[1]{\left\langle #1 \right\rangle}

\newcommand{\bra}[1]{\left\langle #1 \right\rvert}
\newcommand{\ket}[1]{\left\lvert #1 \right\rangle}
\newcommand{\braket}[2]{\left\langle #1 \left.\right\lvert #2 \right\rangle}
\newcommand{\ketbra}[2]{\left\lvert #1 \right\rangle\!\!\left\langle #2 \right\lvert}

\newcommand{\tens}[1]{\stackrel{\leftrightarrow}{#1}}

\newcommand{\jand}{\quad\text{and}\quad}
\newcommand{\jor}{\quad\text{or}\quad}
\newcommand{\for}{\quad\text{for }\,}
\newcommand{\comma}{,\quad}

\newcommand{\ieval}[2]{\Bigg|^{#2}_{#1}}
\newcommand{\deval}[1]{\bigg|_{#1}}

\newcommand{\diff}[2]{\frac{d#1}{d#2}}
\newcommand{\pdiff}[2]{\frac{\partial#1}{\partial#2}}

\newcommand{\id}{\text{id}}

\renewcommand{\setminus}{\smallsetminus}

\newcommand{\customfootnotetext}[2]
{{% Group to localize change to footnote
  \renewcommand{\thefootnote}{#1}% Update footnote counter representation
  \footnotetext[0]{#2}}}

\theoremstyle{plain}

\newtheorem{theorem}{Theorem}
\newtheorem{lemma}[theorem]{Lemma}
\newtheorem{prop}[theorem]{Proposition}
\newtheorem{define}[theorem]{Definition}
\newtheorem{thm}[theorem]{Theorem}
\newtheorem{cor}[theorem]{Corollary}

\numberwithin{equation}{chapter}
\numberwithin{theorem}{chapter}
\numberwithin{figure}{chapter}

\newaliascnt{example}{theorem}
\aliascntresetthe{example}
\crefname{example}{example}{examples}
\newenvironment{ex}[0]{
	\refstepcounter{example}
	\par\noindent
    \textbf{Example \theexample.}\ignorespaces%
}{
	\hfill$\spadesuit$
    \par
}

\newaliascnt{remark}{theorem}
\aliascntresetthe{remark}
\crefname{remark}{remark}{remarks}
\newenvironment{remark}[0]{
	\refstepcounter{remark}
	\par\noindent
    \textbf{Remark \theremark.}\ignorespaces%
}{
	\hfill$\spadesuit$
	\par
}

\newcounter{exercise}
\newenvironment{exercise}[1][\unskip]{
	\refstepcounter{exercise}%
	\par\noindent
    \textbf{Exercise \theexercise.\textsuperscript{#1}}\ignorespaces%
}
{\par}
\numberwithin{exercise}{chapter}

\newcommand{\exercisesection}{
	\section{Exercises\textsuperscript{$\ddag$}}
	\customfootnotetext{$\ddag$}{Difficult exercises will be identified with $\dagger$; Exercises that require optional sections are identified with $\star$}
}

\newcommand{\optionalsection}[1]{
	\section{#1\textsuperscript{$\S$}}
	\customfootnotetext{$\S$}{Optional section: may be skipped without loss of continuity.}
}

\newenvironment{block}
{
    \begin{center}
    \begin{em}\small
    \begin{tabular}{p{0.9\linewidth}}
}
{
    \end{tabular}
    \end{em}
    \end{center}
}

\crefname{exercise}{exercise}{exercises}
\crefname{ex}{example}{examples}
\crefname{figure}{figure}{figures}

\allowdisplaybreaks
