\edef\mychapter{Preface}
\edef\mychapterdate{March 11, 2025}

\chapter*{\mychapter}
As technology propels our world forward, the days of calculating complex lunar trajectories by hand are behind us.
In their place has risen a new era: The era of the computer.
Today, we can entrust the burdens of computation to machines, freeing humans to pursue deeper, more meaningful endeavors.
To reflect our changing world, a revolution in education must revitalize our schools to suit modern times.

Many math courses (especially in America) were primarily designed for an era when computers were insufficient for the computational demands of science.
However, despite significant technological advancements, it seems as if the classroom has largely remained the same, if not stepped in a direction contrary to the needs of the modern world.
In the 1950s to the 1970s, the \textit{New Math} movement began as an effort to dramatically change the way we taught math in schools to reflect the changing times.
However, this revolution was immediately rejected, and in its wake, it seems as if math education went backward in terms of the needs of the modern world with the introduction of Advanced Placement and, eventually, Common Core.
While these were attempts to solve certain problems of education, in turn, it caused education to trend in the wrong way.

Now, more than ever, people feel that the things they learn in school (especially math) provide very little utility in their everyday lives.
But (at least for math), it really doesn't have to be this way.
Math is a subject that goes way beyond memorizing formulas and painstakingly crunching numbers; it's the study of the way humans think and rationalize concepts whose applications go far beyond physics, engineering, etc.
The following text serves as my response to this issue:
A revolution to change the way people perceive math and use it in a way that helps students beyond school.

What the subject of math really captures is a formalization of thought.
Throughout history, math has had a profound impact on the development of human civilization.
From the ancient Greeks to the many modern mathematicians, the goal of math has always been the same: 
To comprehend the uncomprehendable, to break down the complexity of the universe to concepts better suited for human understanding.
This eventually led to the development of numerical computation, but this kind of math represents only a very small portion of the subject.
There are plenty of things one can rationalize that can't be strictly represented numerically.
Without proofs, it's very difficult to capture the true essence of math.
A proof-based approach shines in its ability to demonstrate math's abilities to solve problems that are not strictly numerical.

Math goes far beyond the painstaking computation that we are all taught in schools to the point that when someone brings up mathematics, computation is all that comes to mind.
But this perception of math couldn't be further from the truth.
While most students might find the idea of proofs daunting, I believe this is largely due to their inexperience with them.
My purpose with this book is to bridge that gap since I find that the many skills and lessons (as opposed to the content) of research math can readily be applied to our everyday lives and careers far outside of any immediate application of math.
Whilst students may not go their entire lives remembering the definition of set or function, the mastery of critical thinking will remain for a lifetime...


