\edef\mychapter{Preface}
\edef\mychapterdate{March 11, 2025}

\chapter*{\mychapter}
As technology propels our world forward, the days of calculating complex lunar trajectories by hand are behind us.
In their place has risen a new era: The era of the computer.
Today, we can outsource the burdens of computation to machines, freeing humans to pursue deeper, more meaningful endeavors.
To reflect our changing world, a revolution in education must revitalize our schools to suit the needs of modern times.

Many math courses (especially in America) were primarily designed for an era when computers were insufficient for the computational demands of science.
However, despite significant technological advancements, it seems as if the classroom has largely remained the same, if not stepped in a direction contrary to the needs of the modern world.
Around the 1960s, the \textit{New Math} movement began as an effort to dramatically change how we taught math in schools to reflect the changing times.
However, this revolution was immediately rejected, and in its wake, it seems as if math education took a step in terms of fulfilling the needs of a modern world, as seen with the introduction of Advanced Placement and Common Core.
While these were attempts to solve other problems of education, in turn, they caused education to trend backward.

Now, more than ever, people feel that the things they learn in school (especially math) provide little utility in their everyday lives, and as a training mathematician, I couldn't agree more.
But (at least for math), it really doesn't have to be this way.
Math is a beautiful subject that goes way beyond memorizing formulas and painstakingly crunching numbers; it's the study of the way humans think and rationalize concepts whose applications go far beyond physics, engineering, etc.
The following text serves as my response to this issue:
A revolution to change the way people perceive math and use it in a way that helps students beyond school.

If one pauses to ponder what math truly studies, one might find that at its core, it's the \textit{formal treatment of reason}.
Throughout history, math has had a profound impact on the development of human civilization.
From the ancient Greeks to the many modern mathematicians, the goal of math has always been the same:
To comprehend the incomprehensible by breaking down the complexity of the universe to a collection of formal statements which is then analysed.
Eventually, the formalization of counting and size led to the development of numerical computation, which is often the only paradigm which people associate to math.
While this type of computation is very efficient at solving certain problems, fixation upon this one aspect restricts one to a very small fraction of the entire subject.
As it turns out, math formalizes so much more than just numerics.

This is where proof-based mathematics shines in its ability to look at problems that traditional arithmetic fails at.
As such, without proof, capturing the true essence of math becomes rather difficult.
While most students might find the idea of proofs daunting, I believe this is largely due to their inexperience.
My purpose with this book is to bridge that gap since I find that the many skills and lessons (as opposed to the content) of research math can readily be applied to our everyday lives and careers, far outside any immediate application of math.
Whilst students may not go their entire lives remembering the definitions of sets or functions, through the study of mathematics, the mastery of critical thinking will remain for a lifetime...

This book is adapted from a collection of lecture notes I wrote for a summer precalculus class in 2024.
As such, some content of this book will reflect many standard precalculus curricula throughout America.
However, being that this book develops many standard concepts from the standpoint of a mathematician, the presentation of many topics and ideas may be quite foreign to many students.
As such, included are examples and explanations aimed at motivating and building intuition for the more rigorous constructions.
Each chapter will also start off with the essential concepts and reformulated constructions of standard high school mathematics, but for students who are feeling particularly inspired, an extension is included at the end of each chapter, designed for students who want a glimpse of how these topics extend to more modern ones.

\noindent
-- Tyler Wang (March 2025)


