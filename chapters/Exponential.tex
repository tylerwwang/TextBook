%title
\edef\mychapter{Transcendental Functions}
\edef\mychapterdate{June 25, 2024}

\chapter{\mychapter}
\section{Exponentials}
\subsection{Exponents for $\real$}
This section aims to extend the traditional concept of repeated multiplication to all real numbers.
Immediately, one should notice an issue with this since it isn't really clear what it means to multiply a number by a non-integer number of times.
Thus, to proceed, we need to find a natural way can extend a discrete exponent to a continuous \textit{exponential}.

First, to be clear, when we write $a^n$ for some positive $n$, we are taking the product of $n$ copies of $a$.
We can extend this idea by including negative numbers, where instead we are taking the product of $|n|$ copies of $\frac{1}{a}$.
With this, one should find that the definition $a^0:=1$ is rather natural.
Now, let's observe a few properties of the discrete exponent.

\begin{prop} \label{prop:exponent-prop}
	For any $a \in \real$, the following are true:
	\begin{enumerate}
		\item $a^m a^n = a^{m+n}$, for $m,n \in \nat$.
		\item $(a^m)^n = a^{mn}$, for $m,n \in \nat$.
	\end{enumerate}
\end{prop}
\begin{proof}
	Exercise.
\end{proof}

When constructing the exponential, we wish to accomplish the following things:
\begin{enumerate}
	\item Make sure the result is continuous
	\item To Retain every property as outlined in proposition \eqref{prop:exponent-prop}.
	\item Have the resulting function restricted to the discrete exponent.
\end{enumerate}
In particular, we also would like to make sure this function is unique given some base.
One way to accomplish this is by the following definition.

\begin{define} \label{def:exponential}
	For any $a>0$, a continuous function $\phi:\real \to \real$ is an exponential with base $a$ if the following hold:
	\begin{enumerate}
		\item $\phi(1)=a$, \label{def:exp-prop-unit}
		\item $\phi(x)\cdot [a](y)=[a](x+y)$, \label{def:exp-prop-add}
	\end{enumerate}
\end{define}


From the definition \eqref{def:exponential}, it is not immediately obvious that such a function is unique for each $a$.
In fact, the definition alone doesn't even guarantee this function exists.
Our next goal is to show that for each $a$, an exponential with base $a$ exists and is in fact unique.
To begin, let's examine a few basic properties of any function satisfying the definition \eqref{def:exponential}.

\begin{prop} \label{prop:exp-basic-property}
	Let $\phi$ be any exponential with arbitrary base $a$. The following holds:
	\begin{enumerate}
		\item $\phi(-x)=\frac{1}{\phi(x)}$. \label{prop:exp-prop-neg}
		\item $\phi(0)=1$. \label{prop:exp-prop-unit-preserve}
		\item $(\phi(x))^n=\phi(nx)$ for any $n\in \integ$. \label{prop:exp-prop-multi}
		\item $\phi(x)\neq 0$ for any $x \in \real$. \label{prop:exp-prop-non-zero}
	\end{enumerate}
\end{prop}
\begin{proof}
	By property \eqref{def:exp-prop-add} of the exponential,
	$$\phi(x)\cdot \phi(-x)= \phi(0)=(\phi(1))^0=1.$$
	Thus, we have $\phi(-x)=\frac{1}{\phi(x)}$ by algebraic manipulation, thereby proving statement \eqref{prop:exp-prop-neg}.

	Statement \eqref{prop:exp-prop-unit-preserve} follows by rearranging statement \eqref{prop:exp-prop-neg} as follows:
	$$1=\phi(x)\phi(-x)=\phi(0).$$

	Statement \eqref{prop:exp-prop-multi} for positive integers follows immediately from property \eqref{def:exp-prop-add}.
	Then statements \eqref{prop:exp-prop-neg} and \eqref{prop:exp-prop-unit-preserve} extend this to all integers.


	For statement \eqref{prop:exp-prop-non-zero}, suppose there exists $x \in \real$ such that $\phi(x)=0$.
	Observe
	$$a=\phi(1)=\phi(x)\phi(1-x)=0\cdot \phi(1-x)=0$$
	which is a contradiction since $a \neq 0$.
	Therefore, statement \eqref{prop:exp-prop-non-zero} follows by contradiction.
\end{proof}

In particular statement \eqref{prop:exp-prop-multi} used in combination with property \eqref{def:exp-prop-unit} gives
$$\phi(n)=a^n.$$
Therefore, every exponential with base $a$ restricts to $a^n$.
If such a function satisfying definition \eqref{def:exponential} exists, then this proves that it is in fact an extension of $a^n$.

Our next goal here is to show that every exponential with base $a$ must restrict to a unique function on $\rat$.
The subsequent proofs make heavy use of properties of $n$-roots and $n$-powers, which we will state formally.

\begin{prop} \label{prop:basic-root-prop}
	For any $a>0$, the following hold:
	\begin{enumerate}
		\item $a^n$ is a strictly increasing function if $a>1$
		\item $a^n$ is a strictly decreasing function if $a<1$.
		\item $a^n>n$ for every $n$. In particular, $a^n$ is unbounded from above.
		\item $(\sqrt[m] a)^n = \sqrt[m] {a^n}$, for $m >0 $ and any integer $n$.
		\item $\sqrt[mn] a=\sqrt[m] {\sqrt[n] a}$, for $m,n >0$.
	\end{enumerate}
\end{prop}
\begin{proof}
	Exercise.
\end{proof}

\begin{prop} \label{prop:exp-Q-ext}
	The following holds:
	\begin{enumerate}
		\item If $\phi$ is an exponential with base $a$, $\phi(\frac{1}{n})=\sqrt[n] a$. \label{prop:exp-prop-1-over-n}
		\item Any exponential with base $a$ restricts to the function
			$$[a]_\rat\paren{\frac{p}{q}} = (\sqrt[q] a)^p, \quad \text{for } q>0$$
			on $\rat$. \label{prop:state-Q-ext}
	\end{enumerate}
\end{prop}
\begin{proof}
	To prove \eqref{prop:exp-prop-1-over-n}, observe that by statement \eqref{prop:exp-prop-multi} in proposition \eqref{prop:exp-basic-property} followed by \eqref{def:exp-prop-unit} of the exponential, we have
	$$\paren{\phi\paren{\frac{1}{n}}}^n=\phi(1)= a.$$
	Therefore, we get $\phi\paren{\frac{1}{n}}=\sqrt[n]a$, thus proving \eqref{prop:exp-prop-1-over-n}.

	For \eqref{prop:state-Q-ext}, we can prove this is the unique restriction of any exponential with base $a$ $\phi$ by observing that for $q>0$,
	$$\phi\paren{\frac{p}{q}}=\paren{\phi\paren{\frac{1}{q}}}^p=(\sqrt[q] a)^p$$
	where the middle equality follows from \eqref{prop:exp-prop-multi} of the exponential, and the last equality follows from statement \eqref{prop:exp-prop-1-over-n} in the proposition.

	Thus, it remains to prove $\phi$ is well-defined.
	Suppose $\frac{p}{q}=\frac{p'}{q'}$, then we have $pq'=p'q$.
	This gives us the equation
	$$\sqrt[pq'] a = \sqrt[p'q] a$$
	Taking to the $pp'$ power on both sides yields
	$$(\sqrt[pq'] a)^{pp'} = (\sqrt[q'] a )^{p'} \jand (\sqrt[p'q] a)^{pp'} = (\sqrt[q] a )^{p}.$$
	Therefore, $(\sqrt[q] a )^{p}=(\sqrt[q'] a )^{p'}$ which is exactly the desired result.\footnote{
			A nontrivial fact that we glossed over here is that a (positive) $n$-root indeed always exists for any real number.
			The proof of this fact is not obvious and requires a proper definition of the real numbers.
			However, since this is commonly skipped over (in fact, I doubt many of our readers will even realize this is something that needs proving), I will too.}
\end{proof}

\begin{prop} \label{prop:exp-Q-prop}
	$[a]_\rat$ as defined in proposition \eqref{prop:exp-Q-ext} satisfies the following:
	\begin{enumerate}
		\item $[a]_\rat(r)[a]_\rat(s)=[a]_\rat(r+s)$. \label{prop:exp-Q-add-prop}
		\item $[[a]_\rat(r)]_\rat(s)=[a]_\rat(rs)$. \label{prop:exp-Q-multi-prop}
		\item $[a]_\rat(r)>0$ for all $r \in \rat$. \label{prop:exp-Q-non-zero}
		\item $[a]_\rat$ is strictly increasing for $a>1$, and strictly decreasing for $a<1$. \label{prop:exp-Q-inc-prop}
		\item If $a\neq 1$, for every $x_1>x_0>0$, there exists some $r\in \rat$ such that $x_0 < [a]_\rat(r) < x_1$.\label{prop:exp-Q-connect}
		\item $[a]_\rat$ is continuous.\label{prop:exp-Q-cont-prop}
	\end{enumerate}
\end{prop}
\begin{proof}
	Suppose $r=\frac{m}{n}$ and $s=\frac{p}{q}$ for $n,q>0$.
	Observe $r+s=\frac{mq+np}{nq}$.
	Therefore, statement \eqref{prop:exp-Q-add-prop} follows from
	\begin{align*}
		[a]_\rat(r+s)=(\sqrt[nq] a)^{mq+np}&=(\sqrt[nq] a)^{mq}(\sqrt[nq] a)^{np} \\
		&=(\sqrt[n] a)^m(\sqrt[q] a)^p \\
		&=[a]_\rat(r)[a]_\rat(s)
	\end{align*}
	where the second to last equality follows from proposition \eqref{prop:basic-root-prop}.

	Using the definition of $[a]_\rat$, and proposition \eqref{prop:basic-root-prop}, it follows that
	\begin{align*}
		[[a]_\rat(r)]_\rat(s)
			=\paren{\sqrt[q] {\paren{\sqrt[n] a}^m}}^p
			&=\sqrt[q] { \sqrt[n] {(a^m)^p} } \\
			&=\sqrt[nq] {(a^{mp})} \\
			&=[a]_\rat\paren{\frac{mp}{nq}} \\
			&=[a]_\rat(rs)
	\end{align*}
	which competes the proof for statement \eqref{prop:exp-Q-multi-prop}.

	The proof of statement \eqref{prop:exp-Q-non-zero} is similar to the proof of statement \eqref{prop:exp-prop-non-zero} in proposition \eqref{prop:exp-basic-property}.
	$[a]_\rat$ follows immediately by the definition of $n$-powers and $n$-roots ($n$-roots are always defined to be positive).

	Suppose $a>1$.
	If $r < s$, then $\frac{m}{n}<\frac{p}{q}$.
	Since we assumed $n,q>0$, we have $mq < np$.
	By proposition \eqref{prop:basic-root-prop}, we have $a^{mq} < a^{np}$, since for $a>1$, $a^k$ is a strictly increasing function for integers.
	Taking the $nq$-root on both sides, we obtain $\sqrt[n] {a^m} < \sqrt[q] {a^p}$ since the inverse of an increasing function is increasing.
	$\sqrt[n] {a^m} < \sqrt[q] {a^p}$ is equivalent to $[a]\paren{\frac{m}{n}} < [a] \paren{\frac{p}{q}}$.
	In the case $a<1$, the statement $[a]_\rat$ is decreasing follows from composing $[1/a]_\rat$ with the function $-r$.
	This proves statement \eqref{prop:exp-Q-inc-prop}.

	We show that statement \eqref{prop:exp-Q-connect} is true in the case $a>1$.
	If $x_1>x_0$, we have $\frac{x_1}{x_0}>1$.
	In particular, there exists some $n\in \nat$ such that $a < \paren{\frac{x_1}{x_0}}^n$ (since exponents are unbounded from above and increasing for $a>1$ by proposition \eqref{prop:basic-root-prop}).
	Therefore, ${x_1}^n > a {x_0}^n$.
	Since the set
	$$\{k \in \nat \mid a^k > {x_0}^n\}$$
	is non-empty (since exponents are unbounded from above and increasing for $a>1$) and a subset of $\nat$, a well-ordered set, there exists a least element, which we call $m$.
	Observe $a^{m-1}< {x_0}^n$, therefore
	$${x_1}^n> a{x_0}^n > a \cdot a^{m-1} = a^m > {x_0}^n,$$
	where the last inequality follows from the definition of $m$.
	Since the inverses of strictly increasing functions are increasing and $a^k$ is a strictly increasing function, taking the $n$-root of the previous inequality gives
	$$x_0<\sqrt[n] {a^m} = [a]_\rat\paren{\frac{m}{n}}<x_1.$$
	The case when $a<1$ follows from the $a>1$ case and is left as an exercise.

	To prove continuity, fix any $r_0\in\rat$,\footnote{Remember every point $\rat$ is a limit point $\rat$, so we must check every point.} we prove the limit of $[a]_\rat$ at $r_0$ is exactly $[a]_\rat(r_0)$.
	Fix some $\epsilon>0$.
	Then following statement \eqref{prop:exp-Q-connect} there exists $r_-,r_+$ such that
	$$[a]_\rat(r_-) \in ([a]_\rat(r_0)-\epsilon, [a]_\rat(r_0)) \jand [a]_\rat(r_+)\in ([a]_\rat(r_0), [a]_\rat(r_0) + \epsilon).$$
	Since $[a]_\rat$ is increasing by statement \eqref{prop:exp-Q-inc-prop}, we have $r_- < r_0 < r_+$.
	If we define $\delta:=\min\{r_0-r_-, r_+-r_0\}$, $\delta>0$, and if $|r-r_0|< \delta$,
	then $r_-<r_0-\delta < r < r_0 + \delta < r_+$.
	Since $[a]_\rat$ is increasing, we require
	$$[a]_\rat(r_0)-\epsilon < [a]_\rat(r_-) < [a]_\rat(r)< [a]_\rat (r_+)< [a]_\rat(r_0) + \epsilon.$$
	It follows that
	$$|[a]_\rat(r)-[a]_\rat(r_0)| < \epsilon$$
	thereby proving continuity.\footnote{Special thanks to Marc van Leeuwen for providing this proof. Here's a link to his original post: \href{https://math.stackexchange.com/a/762816}{https://math.stackexchange.com/a/762816}}
\end{proof}

Now to complete our construction by defining $[a]: \real \to \real$ as
\begin{equation}
	[a](x):=\lim_{r \to x} [a]_\rat(r). \label{eq:exp-construction}
\end{equation}
In the subsequent section, we will show this is indeed an exponential with base $a$.
For now, let's observe the fact that every continuous extension of $[a]_\rat$ to a function in $\real$ must satisfy the limit in equation \eqref{eq:exp-construction}.
In addition, since every point in $\real$ is a limit point of $\rat$, the uniqueness of the limit implies $[a](x)$ is the only possible extension.
Since every exponential with base $a$ restricts to $[a]_\rat$ on $\rat$, it follows that if $[a](x)$ is indeed an exponential with base $a$, it is unique.

\begin{prop}
	$[a]$ as given in equation \eqref{eq:exp-construction} is an exponential with base $a$.
\end{prop}
\begin{proof}
	$[a]$ is clearly continuous and $[a](1)=a$ follows immediately from the definition.
	It suffices to prove property \eqref{def:exp-prop-add}.

	For any $x,y\in \real$, since $[a]$ is continuous, the limit
	$$\lim_{u \to x} \lim_{v \to y} [a](u+v)$$
	exists and is equal to
	$$\lim_{r\to x} \lim_{s \to y} [a]_\rat(r+s)$$
	since $\rat$ is dense in $\real$.
	Using this fact, we deduce
	\begin{align*}
		[a](x)[a](y) 	&=\paren{\lim_{r \to x} [a]_\rat(r)}\paren{\lim_{s \to y} [a]_\rat(s)} \\
						&=\lim_{r \to x} \paren{[a]_\rat(r) \paren{\lim_{s \to y} [a]_\rat(s)}} \footnote{This follows from the fact limits commute with multiplication by a scalar, with scalar in this case being $\lim_{s \to y} [a]_\rat(s)$} \\
						&=\lim_{r \to x} \lim_{s \to y} \paren{[a]_\rat(r)[a]_\rat(s)} \\
						&=\lim_{r\to x} \lim_{s \to y}[a]_\rat(r+s) \\
						&=\lim_{r\to x} \lim_{s \to y}[a](r+s) \\
						&=[a](x+y)
	\end{align*}
	where the last equality follows by continuity.
\end{proof}

This shows that the exponential defined in definition \eqref{def:exponential} does indeed exist and is unique.
In addition to the properties listed in \eqref{prop:exp-basic-property}, we have these additional properties, whose proofs we will omit (they follow from the corresponding properties as laid out for $[a]_\rat$ and by density of $\rat$ in $\real$).
\begin{prop} \label{prop:exp-extend-properties}
	For $a >0$, the exponential $[a]$ has the following:
	\begin{enumerate}
		\item $[a]$ is strictly increasing if $a>1$, strictly decreasing if $a<1$ and constant if $a=1$.
		\item $[a]$ is unbounded from above.
		\item $[[a](x)](y)=[a](xy)$.
	\end{enumerate}
\end{prop}

Having constructed the exponential for each $a>0$, let's now focus on a more analytical analysis.
Figure \eqref{fig:exp} displays $[2]$ plotted on a coordinate axis.
We can see that this function is clearly increasing, and, qualitatively, that its range is $(0,\infty)$, a fact we will prove quantitatively in the next section.
We also note that the function gets close to zero as $x \to -\infty$ ($x \to \infty$  in the case $a<1$) without ever reaching it.
This is what's known as an \textit{asymptote}, and in this case, the exponential approaches zero asymptotically as $x \to \infty$ (or $-\infty$).
We will also, from this point on, move to the standard notation for an exponential with base $a$, which is $[a](x):=a^x$.

\begin{figure}[h]
	\centering
	\begin{tikzpicture}
		\begin{axis}
			\addplot[thick, color=blue, samples=100]{2^x};
		\end{axis}
	\end{tikzpicture}
	\caption{}
	\label{fig:exp}
\end{figure}

\subsection{Inverse of Exponential (Logarithm)}
\begin{prop} \label{prop:exp-range}
	For any $a$, the exponential with base $a$ has the range $(0,\infty)$.
\end{prop}
\begin{proof}
	By proposition \eqref{prop:exp-extend-properties}, we have $a^x$ is unbounded from above, hence for every $M>0$, there exists some $x$ such that $M<a^x$.

	Then fix any $N>0$, by the above characterization, there exists some $-x\in \real$ such that $\frac{1}{N} < a^{-x}$.
	Since $\frac{1}{N}$ is positive, we have $\frac{1}{a^{-x}} = a^x < N$.
	To summarize, for every $N>0$, there exists $x\in \real$ such that $a^x<N$.
	We will use this characterization to show $a^x$ maps onto $(0,\infty)$.

	We know $a^x$ maps into $(0,\infty)$ by proposition \eqref{prop:exp-basic-property}, thus it suffices to show this mapping is surjective.
	Fix any $y \in (0,\infty)$.
	Since $a^x$ is unbounded from above, there exists $x_+$ such that $y<a^{x_+}$.
	Using the characterization of the exponential at the beginning of the proof, there exists $x_-$ such that $a^{x_-}<y$.
	By the intermediate value theorem in the previous lesson, there must exist some $w$ between $x_-$ and $x_+$ such that $a^w = y$.
	This completes the proof of the proposition.
\end{proof}

Proposition \eqref{prop:exp-range} proves that the map $a^x:\real \to (0,\infty)$ is bijective since the injectivity of $a^x$ follows from the strictly increasing (or decreasing) nature of $ a^x$.
Therefore, it must admit an inverse of the form $(0,\infty) \to \real$.

\begin{define}[Logarithm]
	Denote the inverse of $a^x$ as $log_a(x)$ read "log base $a$ of $x$".\footnotemark
\end{define}

\begin{figure}[h]
	\centering
	\begin{tikzpicture}
		\begin{axis}
			\addplot[thick, color=blue, domain=-10:10, samples=100]{ln(x)};
		\end{axis}
	\end{tikzpicture}
	\caption{}
	\label{fig:log}
\end{figure}
\footnotetext{When you see a log without a subscript, it can mean one of two things depending on context. In general, this will mean $log_{10}$, since this is often used when plotting data that grows exponentially. In a more mathematical context, this will generally mean natural log, which we will discuss later.}

We can see the plot of the logarithm in \eqref{fig:log}.
Observe, the asymptotic behavior that we observed before slightly changed and in the case of the logarithm, we get $\log_a(x) \to -\infty$ ($\log_a(x) \to \infty$ in the case $a<1$) as $x \to 0$.

The plot clearly indicates that $\log_a(x)$ is also a continuous function, a non-trivial fact given that not every continuous bijection admits a continuous inverse.
I will leave the analytical proof of this fact as an exercise for our readers.\footnote{
	For interested readers, this fact can be derived from the fact that $a^x$ is an open mapping, which, in the case of real maps, is a map where the image of every open interval is open.
	The fact that $a^x$ is an open map follows immediately from the fact $a^x$ is strictly increasing (decreasing) and by the intermediate value theorem.
	Thus, it remains to show these open mapping conditions imply continuity of the inverse (For a bonus, show the converse is true, namely if a continuous bijection has a continuous inverse, then it's an open mapping).}

Now, let's look at some of its other properties.

\begin{prop}
	Let $x,y\in\real$, and $a\in\real^{+}$ then the following hold:
	\begin{enumerate}
		\item $\log_a(x)+\log_a(y)=\log_a(xy)$
		\item $y\log_a(x)=\log_a(x^y)$
	\end{enumerate}
\end{prop}
\begin{proof}
	Since
	$$a^{\log_a(xy)}=xy=a^{\log_a(x)}a^{\log_a(y)}=a^{\log_a(x)+\log_a(y)}.$$
	Then taking the logarithm of both sides gives
	$$\log_a(a^{\log_a(xy)})=\log_a(a^{\log_a(x)+\log_a(y)})$$
	$$\log_a(xy)=\log_a(x)+\log_a(y),$$
	hence proving (1).

	Then since
	$$a^{y\log_a(x)}=x^y=a^{\log(x^y)}$$
	Taking the logarithm of both sides:
	$$\log_a(a^{y\log_a(x)})=\log_a(a^{\log(x^y)})$$
	$$y\log_a(x)=\log(x^y),$$
	hence proving (2).
\end{proof}

\subsection{Change of Basis}
\begin{theorem}
Let $a,b\in(0,\infty)$ and $x\in\real$. Then
$$a^x=b^{\log_b(a)x}$$
\label{thm:changeofexp}
\end{theorem}
\begin{proof}
	Since $a=b^{\log_b(a)}$, using the properties the exponential,
	$$a^x=(b^{\log_b(a)})^x=b^{\log_b(a)x}$$
	hence proving the theorem.
\end{proof}

\begin{theorem}
	Let $a,b\in(0,\infty)$ and $x\in\real$. Then
	$$\log_a(x)=\frac{\log_b(x)}{\log_b(a)}$$

	\label{thm:changeoflog}
\end{theorem}
\begin{proof}
	By the definition of the logarithm, $\log_a(x)$ implies
	$$a^{\log_a(x)}=b^{\log_b(a)\log_a(x)}=x$$
	since $a=b^{\log_b(a)}$.
	Taking the $\log_b$ on both sides gives
	$$\log_b(b^{\log_b(a)\log_a(x)})=\log_b(x)$$
	hence
	$$\log_b(a)\log_a(x)=\log_b(x)$$
	implying
	$$\log_a(x)=\frac{\log_b(x)}{\log_b(a)}$$
\end{proof}

What theorems \eqref{thm:changeofexp} and \eqref{thm:changeoflog} tell us is that we can write any exponential as an exponential function of any other base.
The utility of theorems \eqref{thm:changeofexp} and \eqref{thm:changeoflog} might not be immediately clear, so let's discuss it here.

Like $\pi$, there is an irrational number $e\approx 2.71828$ that we often use in the context of exponentials.
Mathematicians often like writing exponentials in terms of this base due to its special properties, which our readers will encounter in calculus.
We won't cover the special properties of this number because they are closely related to calculus, but we'll introduce the notation to help you understand the concept.
I will reserve the discussion of these properties for when our readers inevitably take calculus (or not).

Using theorem \eqref{thm:changeofexp}, we can write the function
$$f(x)=2^x$$
as
$$f(x)=e^{\log_e(x)x}$$
Because of how fundamental $e^x$ is in the context of exponentials, we give it a special name: $\exp(x)$ (where exp stands for exponential).
The advantage of this notation is that sometimes, with longer arguments, it's a little easier to read, but it also highlights that although similar, the $\real$ exponential and the discrete version are distinct mathematical operations.
Additionally, we also define a special function for $\log_e(x)$ called $\ln(x)$, or natural logarithm (or \textit{logarithme naturel}  in French).

\section{Trigonometry}
 Now we have come to everyone's favorite topic, trigonometry.
 I've included exponentials and trigonometry in the same lesson since they are actually inherently connected.
 In the next lesson, we will see a formula that relates these seemingly distant concepts in a remarkably beautiful way.
\subsection{Definition}
\begin{define} \label{def:trigdefine}
	Using the figure below, let $\odot\,O$ be a unit circle (radius 1), then for any $\theta\in\real$ (measuring counterclockwise from $+x$ axis if $\theta$ is positive, and clockwise if $\theta$ is negative)
	$$x=\cos(\theta) \jand y=\sin(\theta)\footnotemark$$
	\begin{center}
		\begin{tikzpicture}
 			\draw[thick,black] (0,0) circle (2);
 			\filldraw[thick,black] (0,0) coordinate (O) circle (0.05) node[above left] {$O$};

 			\filldraw (1.41,1.41) coordinate (A) circle (0.05) node[right] {$(x,y)$};

 			\draw[thick,blue,->] (0,0) -- (2.5,2.5) ;

 			\draw[thick,<->] (-3,0) -- (3,0) coordinate (B) node[above] {$+x$};
 			\draw[thick,<->] (0,-3) -- (0,3) node[right] {$+y$};

 			\draw pic["$\theta$", draw,->, black, thick, angle radius=1cm] {angle = B--O--A};
		\end{tikzpicture}
	\end{center}
\end{define}

\footnotetext{Check this is the same definition traditional definition, colloquially known as \textit{"SOHCAHTOA"}.}

From the definition, since the values of $\sin(x)$ and $\cos(x)$ vary continuously along the unit circle, they are therefore continuous.
We will leave the rigorous proof of the fact to the end of the section.

Then, using definition \eqref{def:trigdefine}, we can define the other trig functions in the conventional way.
$$\tan(x)=\frac{\sin(x)}{\cos(x)}
\for x\in \cbrak{x\in\real: x\neq \frac{\pi}{2}+\pi n, \:
\forall n\in\nat}$$
$$\cot(x)=\frac{\cos(x)}{\sin(x)}
\for \cbrak{x\in\real :x\neq\pi n, \:
\forall n\in\nat}$$
$$\csc(x)=\frac{1}{\sin(x)} \for \cbrak{x\in\real :x\neq\pi n, \:
\forall n\in\nat}$$
$$\text{sec}(x)=\frac{1}{\cos(x)}
\for x\in \cbrak{x\in\real: x\neq \frac{\pi}{2}+\pi n, \:
\forall n\in\nat}$$

Then to find the image, immediately from the definition \eqref{def:trigdefine}, $|\sin(x)|,|\cos(x)|\le 1$.
Since the values of $-1$ and $1$ are obtained on $\sin(x)$ and $\cos(x)$, it follows from the intermediate value theorem that the range of $\sin(x)$ and $\cos(x)$ is $[-1,1]$.
Since $\csc(x)$ and $\text{sec}(x)$ are just the reciprocals of $\sin(x)$ and $\cos(x)$ respectively, immediately we get the range of these functions is the compound interval $[-\infty,-1]\cup[1,\infty]$.

Finding the image of $\tan(x)$ and $\cot(x)$ can be a bit more tricky, but we will leverage a computer to help us do this. Using figures \eqref{fig:tan} and \eqref{fig:cot}, we see the image of our two functions is $\real$.
\begin{figure}
	\centering
	\includegraphics[scale=0.6]{chapters/assets/exp/tanplot.png}
	\caption{Plot of $\tan(x)$}
	\label{fig:tan}
\end{figure}
\begin{figure}
	\centering
	\includegraphics[scale=0.6]{chapters/assets/exp/cotplot.png}
	\caption{Plot of $\cot(x)$}
	\label{fig:cot}
\end{figure}

\subsection{Basic Identities}
\subsubsection{Trig Function Identities}
\begin{theorem}
\label{thm:perod}
$\sin(x)$ and $\cos(x)$ are $2\pi$ periodic, or equivalently,
$$\sin(x)=\sin(x+2\pi) \jand \cos(x)=\cos(x+2\pi)$$
\end{theorem}
\begin{proof}
	Since $\sin(x)$ and $\cos(x)$ only depend on the angle between the $+x$-axis and a particular ray, by the geometric fact
	$$x \simeq x+2\pi k \for k\in\integ$$
	implies
	$$\sin(x+2\pi k)=\sin(x) \jand
	\cos(x+2\pi k)=\cos(x)$$
	hence proving the theorem.
\end{proof}

\begin{cor}
	$\text{sec}(x)$, $\csc(x)$ are $2\pi$ periodic.
\end{cor}
\begin{proof}
	This proof is trivial with theorem \eqref{thm:perod}.
\end{proof}

\begin{theorem}
$\tan(x)$, $\cot(x)$ are $\pi$ periodic
\label{thm:tanperodic}
\end{theorem}
\begin{proof}
	Postponed for a later section.
\end{proof}

\begin{theorem}
	$\sin(x)$ is odd.
	\label{thm:sinodd}
\end{theorem}
\begin{proof}
	Let's prove this statement first,  only for the domain $[-\pi,\pi]$.
	If $\theta \in [-\pi,\pi]$, negating $\theta$ corresponds to a reflection of over the $x$-axis.
	Therefore, definition \eqref{def:trigdefine} gives
	$$\sin(x)=-\sin(-x).$$

	By theorem \eqref{thm:perod},
	$$\sin(x+2\pi k)=\sin(x)=-\sin(-x)=-\sin(-x+2\pi m).$$
	If we let $m = -k$, we obtain
	\begin{align*}
		\sin(x+2\pi k)&=-\sin(-x-2\pi k) \\
		\sin(x+2\pi k)&=-\sin(-(x+2\pi k)).
	\end{align*}
	Since $x+2\pi k$ covers the entire set $\real$, we conclude that $\sin(x)$ is odd over $\real$.
\end{proof}

\begin{theorem}
	$\cos(x)$ is even.
	\label{thm:coseven}
\end{theorem}
\begin{proof}
	Proof similar to theorem \eqref{thm:sinodd}.
\end{proof}

\begin{cor}
	$\tan(x)$,$\cot(x)$,$\csc(x)$ are odd, and $\text{sec}(x)$ is even.
\end{cor}
\begin{proof}
	Exercise.
\end{proof}


\subsubsection{Pythagorean-esque Identities}
\begin{theorem}[Pythagorean Identity]
\label{thm:pydi}
Let $x\in\real$, then
$$\cos^2(x)+\sin^2(x)=1$$
\end{theorem}
\begin{proof}
	By definition \eqref{def:trigdefine}, the ordered pair $(\cos(x),\sin(x))$ lies on a unit circle. Therefore, since a circle satisfies the relation
	$$x^2+y^2=1$$
	it immediately follows that
	$$\cos^2(x)+\sin^2(x)=1$$
\end{proof}
\begin{cor}
Let $x\in \cbrak{x\in\real: x\neq \frac{\pi}{2}+\pi n, \:
\forall n\in\nat}$, then
$$\text{sec}^2(x)=1+\tan^2(x)$$
\end{cor}
\begin{proof}
	Using theorem \eqref{thm:pydi},
	$$\cos^2(x)+\sin^2(x)=1$$
	Since $\cos(x)\neq0$ by our domain restriction,
	$$1+\frac{\sin^2(x)}{\cos^2(x)}=\frac{1}{\cos^2(x)}=\text{sec}^2(x)$$
	hence
	$$1+\tan^2(x)=\text{sec}^2(x).$$
\end{proof}

\begin{cor}
	Let $x\in \cbrak{x\in\real: x\neq \pi n, \:
\forall n\in\nat}$, then
$$\csc^2(x)=1+\cot^2(x)$$
\end{cor}
\begin{proof}
	Using theorem \eqref{thm:pydi},
	$$\cos^2(x)+\sin^2(x)=1$$
	Since $\sin(x)\neq0$ by our domain restriction,
	$$\frac{\cos^2(x)}{\sin^2(x)}+1=\frac{1}{\sin^2(x)}=\csc^2(x)$$
	hence
	$$1+\cot^2(x)=\csc^2(x).$$
\end{proof}

\subsubsection{Angle Identities}
\begin{theorem}
\label{thm:sumcos}
For $\alpha,\beta\in\real$,
$$\cos(\alpha+\beta)=\cos(\alpha)\cos(\beta)-\sin(\alpha)\sin(\beta)$$
\end{theorem}
\begin{proof}
	Postponed indefinitely.
\end{proof}
\begin{cor}
	For $\alpha,\beta\in\real$,
$$\cos(\alpha-\beta)=\cos(\alpha)\cos(\beta)+\sin(\alpha)\sin(\beta)$$
\end{cor}
\begin{proof}
	Using theorems \eqref{thm:coseven} and \eqref{thm:sinodd},
	$$\cos(\alpha-\beta)=
	\cos(\alpha)\cos(-\beta)-\sin(\alpha)\sin(-\beta)=
	\cos(\alpha)\cos(\beta)+\sin(\alpha)\sin(\beta).$$
\end{proof}

\begin{theorem}
\label{thm:sumsin}
For $\alpha,\beta\in\real$,
$$\sin(\alpha+\beta)=\sin(\alpha)\cos(\beta)+\cos(\alpha)\sin(\beta)$$
\end{theorem}
\begin{proof}
	Postponed indefinitely.
\end{proof}

\begin{cor}
	For $\alpha,\beta\in\real$,
$$\sin(\alpha-\beta)=\sin(\alpha)\cos(\beta)-\cos(\alpha)\sin(\beta)$$
\end{cor}
\begin{proof}
	Using theorems \eqref{thm:coseven} and \eqref{thm:sinodd},
	$$\sin(\alpha-\beta)=\sin(\alpha)\cos(-\beta)+\cos(\alpha)\sin(-\beta)
	=\sin(\alpha)\cos(\beta)-\cos(\beta)\sin(\beta)$$
\end{proof}

\begin{cor}[Double angle]
	For $\theta\in\real$, the following are true:
	\begin{align*}
		\sin(2\theta)&=2\sin(\theta)\cos(\theta) \\
		\cos(2\theta)&=\cos^2(\theta)-\sin^2(\theta)
	\end{align*}

	\label{cor:doubleangle}
\end{cor}
\begin{proof}
	Trivial by theorems \eqref{thm:sumcos} and \eqref{thm:sumsin}.
\end{proof}

\begin{cor}[Half angle]
	For $\theta\in\real$, the following are true:
	$$\sin\paren{\frac{\theta}{2}}=\pm\sqrt{\frac{1-\cos(\theta)}{2}}$$
	$$\cos\paren{\frac{\theta}{2}}=\pm\sqrt{\frac{1+\cos(\theta)}{2}}$$
\end{cor}
\begin{proof}
	Using corollary \eqref{cor:doubleangle} and letting $x=\frac{\theta}{2}$ gives
	\begin{equation} \label{eq:half-angle-eq1}
		\cos(x)=\cos^2\paren{\frac{x}{2}}-\sin^2\paren{\frac{x}{2}}.
	\end{equation}
	Then by theorem \eqref{thm:pydi},
	\begin{align*}
		\cos^2(\frac{x}{2})+\sin^2(\frac{x}{2})	&=1 \\
		\sin^2\paren{\frac{x}{2}}				&=1-\cos^2\paren{\frac{x}{2}}.
	\end{align*}
	Therefore, by substitution into equation \eqref{eq:half-angle-eq1},
	$$\cos(x)=\cos^2\paren{\frac{x}{2}}-\paren{1-\cos^2\paren{\frac{x}{2}}}.$$
	Thus,
	\begin{align*}
		\cos(x)+1				&=2\cos^2\paren{\frac{x}{2}} \\
		\cos\paren{\frac{x}{2}}	&=\pm\sqrt{\frac{\cos(x)+1}{2}}
	\end{align*}
	Then again by theorem \eqref{thm:pydi},
	$$\cos^2\paren{\frac{x}{2}}=1-\sin^2\paren{\frac{x}{2}}.$$
	By substitution into equation \eqref{eq:half-angle-eq1}
	\begin{align*}
		\cos(x)					&=\paren{1-\sin^2\paren{\frac{x}{2}}}-\sin\paren{\frac{x}{2}} \\
		\cos(x)-1				&=-2\sin\paren{\frac{x}{2}} \\
		\sin\paren{\frac{x}{2}}	&=\pm\sqrt{\frac{1-\cos(x)}{2}}
	\end{align*}
	thereby proving the theorem.
\end{proof}

Then using the sum and difference identities for $\sin(x)$ and $\cos(x)$, we can prove the following.

\begin{theorem}
For any $x\in\real$,
$$\sin(x)=\cos\paren{\frac{\pi}{2}-x}$$
\label{thm:coscomp}
\end{theorem}
\begin{proof}
	Using theorem \eqref{thm:sumcos},
	$$\cos\paren{\frac{\pi}{2}-x}=\cos\paren{\frac{\pi}{2}}\cos(x)+\sin\paren{\frac{\pi}{2}}\sin(x)$$
	Since
	$$\sin\paren{\frac{\pi}{2}}=1 \jand \cos\paren{\frac{\pi}{2}}=0,$$
	it immediately follows that
	\begin{align*}
		\cos\paren{\frac{\pi}{2}-x}	&=\cos(x)\cdot 0 + \sin(x)\cdot 1 \\
									&=\sin(x)
	\end{align*}
	hence proving the theorem.
\end{proof}

\begin{cor}
	For any $x\in\real$,
	$$\cos(x)=\sin\paren{\frac{\pi}{2}-x}$$
	\label{cor:sincomp}
\end{cor}
\begin{proof}
	Let $\theta=\frac{\pi}{2}-x$. Then using theorem \eqref{thm:coscomp},
	$$\sin(x)=\cos\paren{\frac{\pi}{2}-x}$$
	Substitution $x$ for $\theta$ gives
	$$\sin\paren{\frac{\pi}{2}-\theta}=\cos(\theta)$$
	hence proving the theorem.
\end{proof}

\begin{cor}
For any $x\in\mathcal{D}$, where $\mathcal{D}$ is the domain for which both $\tan(x)$ and $\cot(x)$ is defined, the following are true:
$$\cot(x)=\tan\paren{\frac{\pi}{2}-x}$$
$$\tan(x)=\cot\paren{\frac{\pi}{2}-x}$$
\end{cor}
\begin{proof}
	Using the definition of $\cot(x)$ and $\tan(x)$:
	$$\tan\paren{\frac{\pi}{2}-x}=\frac{\sin\paren{\frac{\pi}{2}-x}}{\cos\paren{\frac{\pi}{2}-x}}$$
	Then by theorem \eqref{thm:coscomp} and corollary \eqref{cor:sincomp},
	$$\frac{\sin\paren{\frac{\pi}{2}-x}}{\cos\paren{\frac{\pi}{2}-x}}
	=\frac{\cos(x)}{\sin(x)}=\cot(x).$$
	Then $\tan(x)=\cot\paren{\frac{\pi}{2}-x}$ follows immediately.
\end{proof}

\begin{cor}
For any $x\in\mathcal{D}$, where $\mathcal{D}$ is the domain for which both $\csc(x)$ and $\text{sec}(x)$ is defined, the following are true:
$$\csc(x)=\text{sec}\paren{\frac{\pi}{2}-x}$$
$$\text{sec}(x)=\csc\paren{\frac{\pi}{2}-x}$$
\end{cor}
\begin{proof}
	Using the definition of $\csc(x)$ and $\text{sec}(x)$:
	$$\text{sec}\paren{\frac{\pi}{2}-x}=\frac{1}{\cos\paren{\frac{\pi}{2}-x}}$$
	Then by theorem \eqref{thm:coscomp},
	$$\frac{1}{\cos\paren{\frac{\pi}{2}-x}}
	=\frac{\sin(x)}{\sin(x)}=\csc(x)$$
	$\text{sec}(x)=\csc\paren{\frac{\pi}{2}-x}$ follows immediately.
\end{proof}

\subsection{Proving Trig Identities}
Now with these theorems, let's leverage them to prove a couple of identities in a few examples.

\begin{ex}
	Let's verify
	$$\cos^2(x)-\tan^2(x)=2-\sin^2(x)-\text{sec}^2(x)$$
	These types of problems are generally easiest if we establish which side we want to attack first.
	This is usually arbitrary, but for our purposes, I'm going to choose the left side. Using our Pythagorean-esque identities, we establish
	$$\cos^2(x)=1-\sin^2(x) \and
	\tan^2(x)=\text{sec}^2(x)-1$$
	Then by substitution, we get
	$$\cos^2(x)-\tan^2(x)=1-\sin^2(x)-(\text{sec}^2(x)-1)
	=2-\sin^2(x)-\text{sec}^2$$
	hence verifying our desired identity.
\end{ex}

\begin{ex}
	Let's verify
	$$\tan(\alpha+\beta)=\frac{\tan(\alpha)+\tan{\beta}}{1+\tan(\alpha)\tan(\beta)}$$

	Using the theorems \eqref{thm:sumcos} and \eqref{thm:sumsin} and the definition of $\tan(x)$,
	$$\tan(\alpha+\beta)=\frac{\sin(\alpha+\beta)}{\cos(\alpha+\beta)}
	=\frac{\sin(\alpha)\cos(\beta)+\cos(\alpha)\sin(\beta)}
	{\cos(\alpha)\cos(\beta)-\sin(\alpha)\sin(\beta)}.$$
	Then multiplying by
	$$\frac{\frac{1}{\cos(\alpha)\cos(\beta)}}{\frac{1}{\cos(\alpha)\cos(\beta)}}$$
	gives:
	$$=\frac{\sin(\alpha)\cos(\beta)+\cos(\alpha)\sin(\beta)}
	{\cos(\alpha)\cos(\beta)-\sin(\alpha)\sin(\beta)}\cdot
	\frac{\frac{1}{\cos(\alpha)\cos(\beta)}}{\frac{1}{\cos(\alpha)\cos(\beta)}}$$
	$$=\frac{\frac{\sin(\alpha)}{\cos(\alpha}+\frac{\sin(\beta)}{\cos(\beta)}}
	{1-\frac{\sin(\alpha)}{\cos(\alpha)}\cdot\frac{\sin(\beta)}{\cos(\beta)}}$$
	$$=\frac{\tan(\alpha)+\tan{\beta}}{1+\tan(\alpha)\tan(\beta)}$$
	hence verifying our identity.
	\label{ex:sumtan}
\end{ex}

Using the identity that we proved in exercise \eqref{ex:sumtan}, we actually can go back and prove theorem \eqref{thm:tanperodic}.
\begin{proof}
	To prove the theorem, let's prove $\tan(x)=\tan(x+\pi)$.
	Using the identity in exercise \eqref{ex:sumtan},
	$$\tan(x+\pi)=
	\frac{\tan(x)+\tan{\pi}}{1+\tan(x)\tan(\pi)}.$$
	Since $\tan(\pi)=\frac{\sin(\pi)}{\cos(\pi)}=0$,
	$$=\frac{\tan(x)+0}{1+\tan(x)\cdot 0}$$
	$$=\tan(x)$$
	hence $\tan(x)=\tan(x+\pi)$ proving $\tan(x)$ is $\pi$ periodic. Then $\cot(x)$ is $\pi$ periodic follows trivially.
\end{proof}

\begin{ex} \label{ex:trig-cont-ident}
	Using the sum and difference of sin identities, we have
	\begin{align*}
		\sin(x)	&=\sin\paren{\frac{x+y}{2}+\frac{x-y}{2}} \\
				&=\cos\paren{\frac{x+y}{2}}\sin\paren{\frac{x-y}{2}}+\sin\paren{\frac{x+y}{2}}\cos\paren{\frac{x-y}{2}} \\
		\sin(y)	&=\sin\paren{\frac{x+y}{2}-\frac{x-y}{2}} \\
				&=-\cos\paren{\frac{x+y}{2}}\sin\paren{\frac{x-y}{2}}+\sin\paren{\frac{x+y}{2}}\cos\paren{\frac{x-y}{2}}
	\end{align*}
	Subtracting these identities gives us the helpful identity
	$$\sin(x)-\sin(y)=2\cos\paren{\frac{x+y}{2}}\sin\paren{\frac{x-y}{2}}.$$
\end{ex}

\begin{figure}
\centering
	\begin{tikzpicture}
 		\draw[thick,black] (0,0) circle (2);
 		\filldraw[thick, black] (0,0) coordinate (O) circle (0.05) node[above left] {$O$};

 		\filldraw (1.41,1.41) coordinate (A) circle (0.05) node[above, shift={(0,0.2)}] {$A$};
 		\node[below, shift={(0.2,-0.1)}] at (2,0) {$B$};
 		\node[below, shift={(0,-0.1)}] at (1.41,0) {$C$};

 		\draw[thick, black, dashed] (A) -- (1.41,0);
 		\draw[thick, purple, thick] (A) -- (2,0);

 		\draw[thick,blue,->] (0,0) -- (2.5,2.5) ;

 		\draw[thick,<->] (-3,0) -- (3,0) coordinate (B) node[above] {$+x$};
 		\draw[thick,<->] (0,-3) -- (0,3) node[right] {$+y$};

 		\draw pic["$\theta$", draw,->, black, thick, angle radius=1cm] {angle = B--O--A};
	\end{tikzpicture}
	\caption{}
	\label{fig:trig-estimate-plot}
\end{figure}


This identity, shown in example \eqref{ex:trig-cont-ident}, will in fact allow us to prove that the $\sin(x)$ is continuous.
To do so, we still need an additional estimate of $\sin(x)$ derived from its geometric properties.

Let $\theta \in \sbrak{0, \frac{\pi}{2}}$.
In figure \eqref{fig:trig-estimate-plot}, given that $\odot \, O$ is the unit circle, then the length of $\overline {AC}$ is $\sin\theta$.
Therefore, the area of $\triangle AOB$ is simply $\frac{1}{2}\overline{AC}$.
Similarly, observe the sector $AOB$ has area $\frac{\theta}{2}$.
Since the area of the sector completely encloses the area of the triangle, we have
$$\sin\theta \le \theta$$
after multiplying both areas by two.
Since
$$\sin\theta \le 1 \le \frac{\pi}{2},$$
for $\theta \ge \frac{\pi}{2}$, we have the same estimate.
$\sin\theta$ is odd, thus we have
$$\theta \le \sin\theta$$
for when $\theta < 0$.
All together, we have the estimate
$$|\sin\theta| \le \theta.$$
Now we are ready to prove continuity.

\begin{theorem}
	$\sin(x)$ are continuous functions.
\end{theorem}
\begin{proof}
	We show $\lim_{x \to x_0} \sin(x) = \sin(x_0)$.
	Fix any $\epsilon>0$, define $\delta:= \epsilon$.
	For $|x-x_0|<\delta$ we have
	\begin{align*}
		|\sin x-\sin x_0|	&=\abs{2\cos\paren{\frac{x+x_0}{2}}\sin\paren{\frac{x-x_0}{2}}} \\
		&\le 2\abs{\sin\paren{\frac{x-x_0}{2}}} \le 2\paren{\frac{x-x_0}{2}} \\
		&\le \delta = \epsilon.
	\end{align*}
\end{proof}

\begin{cor}
	$\cos(x)$ is continuous.
\end{cor}
\begin{proof}
	Immediate from the formula $\cos(x) = \sin\paren{\frac{\pi}{2}-x}$ from corollary \eqref{cor:sincomp}.
\end{proof}
\begin{cor}
	All other trigonometric functions are continuous on their respective domains.
\end{cor}
\begin{proof}
	Immediate from the definition of these functions and properties of the limit operation.
\end{proof}
