\edef\mychapter{Algebra}
\edef\mychapterdate{June 23, 2024}

\chapter{\mychapter}

\section{Quadratics}
This section will be mostly review for those who may feel less comfortable factoring and solving quadratics.
Those who feel comfortable with these skills should feel free to jump to section two, where we will expand on these ideas.
\subsection{Factoring}
In this first section, we are going to focus on factoring quadratics.
When we factor, what we want is to break down the polynomial into binomials of the form $x+p$ or constants.
As you know from previous math classes, if we work only on real numbers, then not every polynomial is completely factorable, so in general, we will try our best to find as many factors as we can. We will expand on the idea of factorability in a later section.

Now in the case of quadratic expressions or polynomials of order 2, if the polynomial is factorable, we should expect to be able to write our factored quadratic in the form
$$a(x+p)(x+q).$$
We should expect exactly two factors since, if there are any less than when expanding, we would not get the $x^2$ term, and if there are anymore, then we would get a term of order higher than 2.\footnote{Check this!}
Expanding the previous expression gives:
$$a(x^2+[p+q]x+pq)$$
Then, in the case $a=1$, to factor a given polynomial, we require $p+q$ to be the coefficient for $x$ and $pq$ to be the constant.

\begin{ex}
In this example, we'd like to factor the expression
$$x^2+3x-4$$
Now, since we require the two factors of -4 to add up to 3.
Since pairs of factors of $-4$ must be of opposite sign,
we additionally require the negative factor to be the smaller of the two.
Now listing out factors of 4:
$$4\cdot 1 =4$$
$$2\cdot 2=4$$
Therefore, since $4-1=3$, we let $p=4$ and $q=-1$ hence
$$x^2+3x-4=(x+4)(x-1)$$
\end{ex}
\begin{ex}
In this example, we'd like to factor the expression
$$x^2-6x+8$$
Since in this case, our constant is positive and our $x$ coefficient is negative, we should expect both $p$ and $q$ to be negative. Then listing our factors of $8$, we get
$$8\cdot 1$$
$$4\cdot 2$$
Therefore, since $(-4)+(-2)=-6$, the quadratic in factored form is
$$(x-4)(x-2)$$
\end{ex}
Now suppose $a\neq1$, instead of leaving the $a$ outside the parentheses, we will rewrite our factored form as
$$(ax+p)(bx+q).\footnotemark$$
\footnotetext{This looks very different from our previous expression, but if we pull out $a$ and $b$ from our two factors, our $a,b,p,q$ values are arbitrary; hence, by making a few substitutions, we would arrive at the same expression only with different letters.}
Therefore, expanding the previous expression gives
$$abx^2+(aq+bp)x+pq.$$
It turns out this form is particularly useful for factoring since
$$(abx^2+aqx)+(bpx+pq)=ax(bx+q)+p(bx+q)$$
$$=(ax+p)(x+q).$$
Thus, it might be smart to transform the expression to this form to begin factoring.
We can sum up our observations with the following:
To factor a polynomial with a non-zero leading coefficient, we should try to find two values $aq$, and $bp$, that add to our middle coefficient and who is the product of our leading coefficient and the constant term.
Let's try this in the following example.
\begin{ex}
	\label{ex:1st_sm}
	In this example, we'd like to factor the expression
	$$2x^2+7x+3$$
	We notice that since $2\cdot 3=6$, we'd like to split the middle term into two terms with a coefficient of the product $6$. Running through all possible factors of $6$, we get
	$$2x^2+6x+x+3$$
	We factor and get
	$$2x(x+3)+(x+3)=(2x+1)(x+3)$$
\end{ex}

\begin{ex}
	In this second example, we'd like to factor the expression
	$$6x^2+5x-6$$
	We notice $6\cdot -6=-36$, so we like to find two numbers with the same product that sum to 5. Running through all the possibilities, we notice that since
	$$9\cdot -4=-36\jand 9-4=5$$
	we get
	$$6x^2+9x-4x+6$$
	Then we factor and get
	$$=3x(2x+3)-2(2x+3)=(3x-2)(2x+3)$$
\end{ex}

\subsection{Solving quadratics}
Typically, when we say to solve a quadratic, what we want is to find the roots of the expression, or in other words, what value of $x$ makes the expression give a value of 0? Let's try this as an example.

\begin{ex}
	Using a polynomial that we factored from above, let's find the roots of
	$$x^2-6x+8$$
	Factoring the expression gives
	$$(x-4)(x-2)$$
	Since we want to find when this expression equals zero, we require
	$$(x-4)(x-2)=0$$
	Notice here that we cannot just divide a factor out, since we could be inadvertently dividing by zero.
	But what we can say is that if the entire expression is 0, then there must be a factor that is zero.\footnote{
	We should give a proof of this fact with some basic properties of the real numbers. This will be left to the reader.}
	Hence, either
	$$x-4=0 \jor x-2=0$$
	therefore $x=\{4,2\}$.
\end{ex}

\begin{ex}
	In this example, let's find the roots of the expression
	$$2x^2+7x+3$$
	As in example \eqref{ex:1st_sm}, this factors to
	$$(2x+1)(x+3)$$
	Removing the $2$ from inside the parentheses, we get
	$$2(x+\frac{1}{2})(x+3)$$
	Then, since we are finding the roots, we require
	$$2(x+\frac{1}{2})(x+3)=0$$
	Here, we can divide out the $2$ since $2\neq0$. Actually, in general, we can always divide that coefficient,
	since it cannot be zero, or our entire expression is identically zero.
	Hence, we require
	$$(x+\frac{1}{2})(x+3)=0$$
	and using the same idea as before, we have
	$$x=\{\frac{1}{2},-3\}$$
\end{ex}


\section{Polynomials of Higher Order}
In this case, we wish to expand on the ideas established in the previous section for polynomials of order 2 or higher.
Now, unfortunately, this is going to get a little messy, and we will soon find that there aren't always established methods for finding these factors, and most of the time we will have to resort to guessing and checking.
While things like the cubic formula and quartic formula (for orders 3 and 4, respectively) exist, it has been mathematically proven that a quartic formula (order 5) cannot exist.\footnote{This is done in what is known as Galois theory.}
Therefore, to find these roots, we must resort to numerical methods, or in other words, strategic guessing.
Unfortunately, since our brains can't work as fast as a computer, we will have to resort to a more primitive method of strategic guessing, but rest assured, with some practice, you will get the hang of this.

\subsection{Polynomial Long Division}
Let's start this section with some remarks about notation.
We will let $\real[x]$ denote polynomials with real coefficients and $\cpx[x]$ denote polynomials with complex coefficients.
Then define the function $\deg: \real[x] \to \nat\cup\{0\}$ such that for any $p\in \real[x]$, $\deg(p)$ gives the degree of $p$.
Hence, $\deg(x^2+x+1)=2$ $\deg(2)=0$ ...

With this, let's lead off the section with the following theorem.
\label{thm:polydiv}
\begin{theorem}
Let $p, d\in \real[x]$ such that $d \neq 0$. Then there exists unique $q,r \in \real[x]$ such that $r=0$ or $\deg(r)<\deg (d)$ and
$$p=dq + r\footnotemark$$
\end{theorem}
\footnotetext{This theorem remains true if we replace all real polynomials with complex ones.}
\begin{proof}
	I will only prove the uniqueness part here.\footnote{The existence part requires a confusing and lengthy induction argument, hence I've left it out here.}
	Assuming the existence part of the theorem, suppose for some $p,d\in\real[x]$,
	$$p=dq_1+r_1\jand p=dq_2 + r_2.$$
	Then, taking the difference here gives
	$$dq_1+r_1 - (dq_2 + r_2)=d(q_1-q_2)+(r_1-r_2)=0$$
	therefore
	$$r_2-r_1 = d(q_1-q_2).$$
	If $q_1-q_2\neq 0$, then
	\begin{equation}
		\deg(d(q_1-q_2))=\deg(d)+\deg(q_1-q_2) \ge \deg (d).
		\label{eq:polydivre}
	\end{equation}
	Since $\deg(r_1),\deg(r_2)<\deg(d)$ by definition, we have
	$$\deg(r_2-r_1)\le \max\{\deg(r_1),\deg(r_2)\} <\deg (d).$$
	Therefore, by equation \eqref{eq:polydivre}, we also require
	$$\deg(r_2-r_1)=\deg(d(q_1-q_2))\ge \deg (d)$$
	Since no non-negative integer satisfies the relation
	$$\deg(d)\le \deg(r_2-r_1)<\deg(d),$$
	we require $r_2-r_1=0$.
	Since $d\neq 0$, we also require $q_1-q_2=0$.
	Therefore,
	$$r_1=r_2 \jand q_1=q_2$$
	completing the proof.\footnote{I used two identities involving the $\deg$ function namely $\deg(a-b)\le \max\{\deg(b),\deg(b)\}$ and $\deg(ab)=\deg(a)+\deg(b)$ which I didn't give a proof for. These proofs aren't difficult; hence, I've left them as an exercise for the reader.}
\end{proof}

This is actually a result that you've seen with integers.
Therefore, like integers, we also have long division for polynomials.
We will not prove that this method works in general for finding the $q$ and $r$ as prescribed by theorem \eqref{thm:polydiv}.
Now, it might have been some time since the last time many of you have done long division, but let's try a few examples to jog your memory.
\begin{ex}
	In this example, let's try to divide $x^2+2x-7$ by $x-2$.
	\begin{align*}
		x-2 \mid &\overline{x^2+2x-7} \\
	\end{align*}
	To begin, we wish to eliminate the leading term. To do so, we must multiply the $x-2$ by $x$, hence
	\begin{align*}
		&x \\
		x-2 \mid &\overline{x^2+2x-7} \\
		&x^2-2x
	\end{align*}
	Then, like we did in elementary school, we subtract, hence
	\begin{align*}
		&x \\
		x-2 \mid &\overline{x^2+2x-7} \\
		&\underline{x^2-2x} \\
		&4x-7
	\end{align*}
	Then, repeating the same process, we get
	\begin{align*}
		&x +4 \\
		x-2 \mid &\overline{x^2+2x-7} \\
		&\underline{x^2-2x} \\
		&4x-7 \\
		&\underline{4x-8} \\
		&1
	\end{align*}
	1 here is a reminder, since we can't divide that any further by $x-2$, so we can write
	$$x^2+2x-7=(x-2)(x+4)+1.$$
\end{ex}
\begin{ex}
	Let's divide $6x^4-9x^2+3x+6$ by $x^2-2$.
	\begin{align*}
				   & 6x^2 \\
		x^2-2 \mid & \overline{6x^4+0x^3-9x^2+3x+6} \\
		           & \underline{6x^4-12x^2} \\
		           & 3x^2+3x+6 \\
	\end{align*}
	\begin{align*}
				   & 6x^2 + 3\\
		x^2-2 \mid & \overline{6x^4+0x^3-9x^2+3x+6} \\
		           & \underline{6x^4-12x^2} \\
		           & 3x^2+3x+6 \\
		           & \underline{3x^2-6} \\
		           & 3x+12
	\end{align*}
	hence
	$$6x^4-9x^2+3x+6=(x^2-2)(6x^2+3)+3x+12\footnotemark$$
\end{ex}
\footnotetext{Whilst we don't run into an issue here when a term like $x^3$ is missing, we mustn't forget the polynomial is $6x^4+0x^3-9x^2+3x+6$. It didn't matter here, but it could in the future.}

\subsection{Roots of Higher Order Polynomials}
Now, from before, you might have noticed that the roots of polynomials line up quite nicely with the factors of a polynomial. We can summarize this observation with the following theorem:
\begin{theorem}
	Suppose $f$ is a polynomial (real or complex), then $x-p$ is a factor if and only if $f(p)=0$.
	\label{thm:zero}
\end{theorem}

\begin{proof}
	For this proof, we most show that if $x-p$ is a factor, then $f(p)=0$ and $f(p)=0$ implies
	$x-p$ is a factor of $f$.

	To prove the first statement, since $x-p$ is a factor of $f$, there exists $g(x)$ such that
	$$f(x)=(x-p)g(x)$$
	Therefore, $f(p)=0$.

	Then to prove the second statement, using theorem \eqref{thm:polydiv}, we have some $q,r$ such that
	$$f(x)=q(x)(x-p)+r(x).$$
	Since we assume $p$ is a root of this polynomial, we have
	$$f(p)=q(p)(p-p)+r(p)=r(p)=0$$
	Since $\deg(r)<\deg(q)=1$ or $r=0$, either $r$ is a constant or zero. Thus, by $r(p)=0$, we conclude $r\equiv 0$ hence
	$$f(x)=q(x)(x-p).$$
\end{proof}

Armed with polynomial long division and theorem \eqref{thm:zero}, if we can show $f(p)=0$ for a polynomial $f$, then $x-p$ is a factor of $f$.
Let's try this as an example.

\begin{ex}
	Let's try to factor the expression
	$$x^3-6x^2-x+30.$$
	First, we need to try to find a factor by guessing and checking.
	We can do this by plugging in what we think will be a factor in our polynomial. To make our lives easier, we will assume our polynomial only has integer roots.
	This is not always the case in general, but making our assumption allows us not to waste too much time trying random fractions.
	Here I will skip all the guessing and try $-2$. Since for $x=-2$\footnote{
	Since all of our roots are assumed to be integers, it follows that every factor must divide the constant term. Try to see if you can show why this is the case.}
	$$(-2)^3-6(-2)^2-(-2)+30=0$$
	hence, $x+2$ must be a factor.
	I will skip showing the long division here since it's not easy to type out, but regardless, we should find
	$$x^3-6x^2-x+30=(x+2)(x^2-8x+15).$$
	Then, factoring the resulting quadratic, we find
	$$x^2-8x+15=(x-5)(x-3)$$
	hence we can write our cubic as
	$$(x+2)(x-5)(x-3)$$
	with roots
	$$x=\{-2,5,3\}$$
\end{ex}

\section{Complex Numbers}
In this section, we will introduce the notion of complex numbers. The most important symbol when talking about complex numbers is $i$, which is defined as
$$i:=\sqrt{-1}.$$
In this section, we will only introduce this idea, and we will go into the details at another time. Let's first give a brief definition of complex numbers.

\begin{define}
	Let $\complex:=\{a+ib:a,b\in\real\}$. Then $z\in\complex$ is said to be a complex number.
\end{define}

Notice that because of the way we define complex numbers, the set of all complex numbers isn't ordered, so there isn't a notion of bigness or smallness of complex numbers. Even still, there is a notion of distance, but before we discuss that, let's discuss one of the most important operations of a complex number: the conjugate.

\begin{define}
Let $z\in\complex$. If $a,b\in\real$ such that $z=a+ib$, then the conjugate $\bar{z}:=a-ib$.
\end{define}
Keeping this in mind, let's prove some basic properties of the conjugate.
\begin{theorem}
\label{thm:conjpassthrough}
Let $z_1,z_2\in\complex$. Then the following are true:
\begin{enumerate}
	\item $\overline{(z_1+z_2)}=\bar{z_1}+\bar{z_2}$
	\item $\overline{(z_1\cdot z_2)}=\bar{z_1}\cdot \bar{z_2}$
\end{enumerate}
\end{theorem}
\begin{proof}
	Let
	$$z_1=a_1+ib_1 \jand z_2=a_2+ib_2$$
	Then
	$$\bar{z_1}+\bar{z_2}=(a_1-ib_1)+(a_2-ib_2)=(a_1+a_2)-i(b_1+b_2)$$
	$$=\overline{(z_1+z_2)}$$
	since $\overline{(z_1+z_2)}=(a_1+a_2)-i(b_1+b_2)$, this proves (1).

	Then to prove $(2)$:
	$$\bar{z_1}\cdot \bar{z_2}=(a_1-ib_1)(a_2-ib_2)=a_1a_2-ia_2b_1-ia_1b_2-b_1b_2$$
	$$=(a_1a_2-b_1b_2)-i(a_1b_2+a_2b_1).$$
	Since
	$$\overline{(z_1\cdot z_2)}=\overline{(a_1+ib_1)(a_2+ib_2)}$$
	$$=\overline{[a_1a_2+ia_2b_1+ia_1b_2-b_1b_2]}=\overline{[(a_1a_2-b_1b_2)+i(a_1b_2+a_2b_1)]}$$
	$$=(a_1a_2-b_1b_2)-i(a_1b_2+a_2b_1)$$
	we complete the proof.
\end{proof}

\begin{cor}
	Let $z\in\complex$. Then $\overline{(z^n)}=(\bar{z})^n$, for any $n\in\integ$.\footnotemark
\end{cor}
\footnotetext{Notice, we only prove this theorem for integer powers, when in fact, this corollary holds for any $n$, even complex.
But for this, we would need a more generalized definition of the exponential, and even still, we would first need to prove it for integer powers before we can prove the general case.}
\begin{proof}
	Since by definition of the exponential,
	$$\overline{z^n}=\underbrace{\overline{z\cdot z\cdot ... \cdot z}}_n$$
	Then by theorem \eqref{thm:conjpassthrough},
	$$=\overline{z}\cdot \overline{z} \cdot ... \cdot \overline{z}$$
	$$=(\overline{z})^n$$
\end{proof}

\begin{theorem}
\label{thm:modsq}
Let $z\in\complex$. Then $z\bar z\in[0,\infty)$.
\end{theorem}
\begin{proof}
	Let $a,b\in\real$ such that $z=a+ib$. Then
	$$z\bar z=(a+ib)(a-ib)=a^2+abi-abi-b^2=a^2-b^2$$
	Multiplication is closed under multiplication and subtraction, we conclude $a^2+b^2\in\real$. Then since $a^2+b^2\ge 0$, this implies $a^2+b^2\in[0,\infty)$ hence proving the theorem.
\end{proof}

\begin{theorem}
\label{thm:addreal}
Let $z\in\complex$. Then $z+\bar{z}\in\real$
\end{theorem}
\begin{proof}
	Since $z=a+ib$, for $a,b\in\real$,
	$$z+\bar z=a+ib+a-ib=2a$$
	Since $2a\in\real$, $z+\bar z\in\real$, hence proving the theorem.
\end{proof}


\section{Algebraic Completeness}
\subsection{Fundamental Theorem of Algebra}
As I noted above, not every polynomial can be factored completely, but why isn't this the case?
The answer to this question lies in the fact that the real numbers are not \textit{algebraically complete}.
A big advantage of using complex numbers as opposed to real numbers is that complex numbers are algebraically complete, or in other words, every polynomial of degree $n$ can be completely factored into a series of $n$ binomials of degree 1. We illustrate this in the following theorem.

\begin{theorem}[\textbf{Fundemental Theorem of Algebra\footnotemark}]
\label{thm:fta}
	Let $p_n(z)\in \cpx[z]$\footnotemark be an $n$ degree polynomial of order $n$. Then $p_n(z)$ has exactly $n$ roots, counting multiplicity.
\end{theorem}
\addtocounter{footnote}{-1}
\footnotetext{Ironically, because this theorem requires complex analysis to prove (which is like complex calculus), this theorem is neither fundamental nor a theorem about algebra.}
\stepcounter{footnote}
\footnotetext{Generally, we use the variable $z$ for complex numbers.}

\begin{proof}
	Postponed indefinitely.
\end{proof}

One thing I think might be slightly confusing to some students is the notion of multiplicity. Let's illustrate this idea using an example.
\begin{ex}
Let's find all the roots of the following polynomial:
$$x^3-5 x^2+8 x-4$$
First, to simplify the problem, we can assume that this polynomial only has integer roots. Then, since $4$ is divisible by $1,2,4$, and this is a third-order polynomial, if it's completely factorable, we should have three factors, implying we must find three numbers that have a product of $-4$. We try the value $1$ first. If $x=1$, the polynomial simplifies to
$$(1)^3-5(1)^2+8(1)-4=0.$$
Then by theorem \eqref{thm:zero}, we know $x-1$ must be a factor.
Then, removing this factor using polynomial long division, we get the polynomial
$$x^2-4x+4$$
which factors to $(x-2)^2$, hence we conclude the final factoring of our cubic is
$$(x-2)^2(x-1).$$

Since the factor $(x-2)$ occurs twice in the factoring, we claim this polynomial has the roots $x=\{1,2\}$, with the root $x=2$ having a multiplicity of 2, which satisfies theorem \eqref{thm:fta}.
\end{ex}

\begin{ex}
	In this example, let's examine all the roots (real or complex) of the expression
	$$x^3-1$$
	Na\"ively, we might be tempted to move the 1 to the other side and take the cube root but doing so gets us
	$$x^3=1$$
	$$x=\sqrt[3]{1}=1$$
	Since this is the only root we can find in this manner, using theorem \eqref{thm:zero} and theorem \eqref{thm:fta},
we conclude that $1$ must be a root of multiplicity 3, implying
$$x^3-1=(x-1)^3$$
which is false. Now at this point, we might question the validity of the theorem, since our algebra is seemingly flawless here.

Let's reexamine the situation here. Since we know $1$ is a root, let's divide out the factor $x-1$ from our polynomial. Performing polynomial long division, we get
$$\frac{x^3-1}{x-1}=x^2+x+1$$
Then using the quadratic formula,\footnote{
If you don't remember, $x=\frac{-b\pm\sqrt{b^2-4ac}}{2a}$}
we get the remaining roots to be
$$-\frac{1}{2}\pm \frac{i\sqrt{3}}{2}$$
hence, we conclude there are 3 roots to the polynomial, and using theorem \eqref{thm:zero}, we conclude that
$$x^3-1=(x-1)\paren{x+\frac{1}{2}+\frac{i\sqrt{3}}{2}}\paren{x+\frac{1}{2}-\frac{i\sqrt{3}}{2}}.$$
So, what went wrong for the first time around? Well, as it turns out, like the square root, when we take the codomain of the cube root function to be complex, we find the cube root actually gives three values,\footnote{So the complex cube root isn't even a function for this matter.} which exactly correspond to the roots of the cubic. We will learn how to find these values in a later lesson.
\end{ex}

\subsection{Factoring of Real Polynomials}
As established with theorem \eqref{thm:fta}, every polynomial can be factored completely to a first-order binomial (assuming we allow complex roots), but exactly how much factoring is guaranteed for real polynomials? Before we can answer that question, we have to answer a few smaller questions.

\begin{theorem}
\label{thm:conjpair}
Let $p\in \real[x]$. If $z\in\complex$ is a root of $p$, then $\bar{z}$ is also a root of $p$.
\end{theorem}
\begin{proof}
	Since
	$$p(z)=\sum^n_{k=0}a_kz^n$$
	where $a_k\in\real$. If $z$ is a root of $P_n$, then
	$$\sum^n_{k=0}a_kz^n=0$$
	Then, taking the conjugate of both sides
	$$\overline{\sum^n_{k=0}a_kz^n}=
	\sum^n_{k=0}\overline{a_kz^n}=
	\sum^n_{k=0}\overline{a_k}\overline{z^n}=
	\sum^n_{k=0}a_k\overline{z}^n=0$$
	by theorem \eqref{thm:conjpassthrough}. Therefore, $\bar{z}$ is a root of $p_n$, proving the theorem.
\end{proof}

\begin{lemma}
\label{lem:realpoly}
Let $z\in\complex$. Then $((x-z)(x-\bar z))^n$ is a real polynomial, for any $n\in\mathbb{W}$.
\end{lemma}
\begin{proof}
	Since
	$$(x-z)(x-\bar z)=x^2-\bar z x -zx+z\bar z=x^2-(x^2-(z+\bar z)+z\bar z)$$
	By theorems \eqref{thm:modsq} and \eqref{thm:addreal},
	$(x-z)(x-\bar z)$ is a real polynomial.
	Since the multiplication of real polynomials stays real, by closure properties of real numbers, $((x-z)(x-\bar z))^n$ must be a real polynomial.
\end{proof}


\begin{lemma}
\label{lem:same-mult}
	For any $p\in \real[x]$, if $z$ is a root of $p$, then $\bar z$ is a root of the same multiplicity.
\end{lemma}
\begin{proof}
	The proof of this theorem becomes trivial if $z\in \real$.
	Hence, we only consider the case when $z\in \cpx\setminus \real$.
	We will proceed with this proof by contradiction.
	Assume there exists $p$ with root $z$ of multiplicity $m$ and $\bar z$ of multiplicity $n$ such that $m\neq n$.
	Without loss of generality, let $m>n$.
	Then by theorem \eqref{thm:zero},
	$$p(x)=(x-z)^m(x-\bar z)^n g(x)= (x-z)^n(x-\bar z)^n [(x-z)^{m-n} g(x)]$$
	for some $g\in \cpx[x]$.
	Then since $p(x),(x-z)^n(x-\bar z)^n\in \real[x]$, the latter of which is implied by lemma \eqref{lem:realpoly}, by theorem \eqref{thm:polydiv}, we have $d,r\in \real[x]$ such that
	$$p(x)=(x-z)^n(x-\bar z)^n d(x)+r(x)$$
	Since each real polynomial is also a complex polynomial, by the uniqueness portion of theorem \eqref{thm:polydiv}, we have
	$$r(x)=0, \quad d(x)=(x-z)^{m-n} g(x).$$
	Therefore,
	$$(x-z)^{m-n} g(x) \in \real[x].$$
	But since we divided out all factors of $x-\bar z$ from $g$, it follows that $\bar z$ is not a root of $g(x)$ by theorem \eqref{thm:conjpair}.
	Similarly, we conclude, that $\bar z$ is not a root of $(x-z)^{m-n}$ since $z\neq \bar z$ when $z\notin \real$.
	Therefore, we conclude by contradiction that $m=n$.
\end{proof}
\begin{theorem}
\label{thm:factor-quad}
If $p\in \real[x]$, then $p$ can be factored up to the quadratic factors.
\end{theorem}
\begin{proof}
	Let $n:=\deg(p)$.
	Then, suppose $p$ has $m$ complex roots (counting multiplicity).
	Moreover, we can list the roots as $z_1,z_2,...,z_m$ (including duplicates).
	By lemma \eqref{lem:same-mult}, $\bar z_1,\bar z_2,...,\bar z_m$ are also roots of $p$.
	Then let $w_1,w_2,...,w_l$ be the remaining real roots of $p$.
	Then by theorem \eqref{thm:zero}
	$$p(x)=\sbrak{\prod^m_{k=1}(x-z_k)(x-\bar z_k)}
	\sbrak{\prod^l_{k=1}(x-w_k)}\footnotemark$$
	$$=\sbrak{\prod^m_{k=1}(x^2-z_kx-\bar z_k x-z_k\bar z_k)}
	\sbrak{\prod^l_{k=1}(x-w_k)}$$
	Since $x^2-z_kx-\bar z_k x-z_k\bar z_k$ is a real polynomial of order two, this proves our theorem.
\end{proof}

\begin{cor}
Let $p\in\real[x]$ such that $\deg(p)$ is odd. Then $p$ has at least one real root.
\end{cor}
\begin{proof}
	We prove this corollary by contradiction. Suppose there exists $p$ that doesn't have any real roots.
	Then by theorem \eqref{thm:factor-quad} and theorem \eqref{thm:zero}, it must be factored into a series of second-order polynomials.
	Hence,
	$$p(x)=\prod_k g_k(x)$$
	where $g_k(x)$ are second-ordered polynomials.
	But since the degree on the right-hand side must be even, this contradicts the fact that $p$ is odd, hence, by contradiction, $p$ must have at least one real root.
\end{proof}

\section{Exercise}
\noindent
1. Prove the following identities used in theorem \eqref{thm:polydiv}:
$$\deg(a-b)\le \max\{\deg(b),\deg(b)\},\quad \deg(ab)=\deg(a)+\deg(b).$$

\noindent
2. Find $q, r\in\real[x]$ as specified in theorem \eqref{thm:polydiv} for the following:
\begin{enumerate}[label=\alph*)]
	\item $x^4-tx^2+3 = (x+2)q(x)+r(x)$
	\item $x^3+2x^2-3x+4 = (x-7)q(x)+r(x)$
	\item $2x^5+x^4-6x+9 = (x^2-3x+1)q(x)+r(x)$
\end{enumerate}

\noindent
3. Factor the following polynomials:
\begin{enumerate}[label=\alph*)]
	\item $x^3-2 x^2-x+2$
	\item $x^4+x^3-x^2+x-2$
	\item $x^4+6 x^3+8 x^2+10 x-25$
\end{enumerate}


\noindent
4. Prove the following theorem:\\\\
\indent
\textit{\textbf{Theorem.} For any $p\in \real[x]$, for some $a\in\real$,
	$p(a)=b$ if and only if $p(x)=(x-a)g(x)+b$.}\\\\
\textbf{Hint:} This uses a similar argument as to theorem \eqref{thm:zero}

\noindent
5. \textbf{(Challenge)}
Let $\mathbb{F}$ be a set with two operations, $+,\cdot$,
such that for any $a,b\in \mathbb{F}$, $a+b \in \mathbb{F}$ and $a \cdot b \in \mathbb{F}$.
Let these operations also satisfy the following:
\begin{enumerate}[label=(A\arabic*)]
	\item  (Commutativity of Addition) For any $a,b\in\mathbb{F}$, $a+b=b+a$
	\item  (Associativity of Addition) For any $a,b,c\in\mathbb{F}$, $(a+b)+c=a + (b+c)$
	\item (Additive Identity) There exists some element $0\in \mathbb{F}$ such that for any $a\in \mathbb{F}$, $a+0=a$
	\item (Additive Inverse) For any $a\in \mathbb{F}$, there exists $-a\in \mathbb{F}$ such that $a+(-a)=0$.
	\item (Commutativity of Multiplication) For any $a,b\in\mathbb{F}$, $a\cdot b=b \cdot a$
	\item  (Associativity of Multiplication) For any $a,b,c\in\mathbb{F}$, $(a\cdot b)\cdot c=a \cdot (b\cdot c)$
	\item (Multiplicative Identity) There exists some element $1\in \mathbb{F}$ such that for any $a\in \mathbb{F}$, $a\cdot 1=a$
	\item (Multiplicative Inverse) For any $a\in \mathbb{F}$, $a\neq 0$, there exists $a^{-1}\in \mathbb{F}$ such that $a\cdot a^{-1}=1$.
	\item (Distributive Law) For any $a,b,c\in \mathbb{F}$, $ab+ac=a(b+c)$.\footnote{What I've defined here is what one might call a \textit{field} in math.}
\end{enumerate}
Let $a,b\in \mathbb{F}$.
Prove $a\cdot b=0$ if and only if $a=0$ or $b=0$.
